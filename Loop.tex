
%Circular Loop  of Constant Current

\section{Circular Loop  of Constant Current}
\subsection{要求}
\noindent 画出不同半径均匀电流环方向图$l=\dfrac{\lambda}{20},\dfrac{\lambda}{6.28},\dfrac{\lambda}{2},0.61\lambda,\lambda,4\lambda$ 

\subsection{原理及推导}
均匀电流环的远场解


\begin{equation}
E_\phi\simeq \dfrac{ak\eta I_0e^{-jkr}}{2r} J_1\left( ka\sin\theta\right)
\end{equation}


\begin{equation}
H_\theta\simeq -\dfrac{E_\phi}{\eta}
\simeq -\dfrac{ak I_0e^{-jkr}}{2r} J_1\left( ka\sin\theta\right)
\end{equation}

由于, 最终需要得到归一化的功率方向图.所以常数项可以直接忽略. 
远场可以视为TEM波,故
\begin{equation}
P=\dfrac{E^2}{\eta} \simeq \left[J_1\left( ka\sin\theta\right)\right]^2
\end{equation}
编程时亦只关注这一部分. 


\subsection{结果与分析}
\subsubsection{方向图}


\subsection{程序}
\noindent \textbf{绘制方向图主程序}
\begin{lstlisting}[language={matlab},keywordstyle=\color{blue!70},commentstyle=\color{red!50!green!50!blue!50},frame=shadowbox, rulesepcolor=\color{red!20!green!20!blue!20}] 

\end{lstlisting}

